% NWQMC.tex
\documentclass{beamer}
\usetheme{Boadilla}
\usepackage{amsmath}


\AtBeginSection[]
{
  \begin{frame}
    \frametitle{Table of Contents}
    \tableofcontents[currentsection]
  \end{frame}
}


% items enclosed in square brackets are optional; explanation below
\title[surprise]{Bayesian surprise as a tool for monitoring sensor networks}
\subtitle[na]{no subtitle yet}
\author[W. Brooks]{Wesley Brooks}
\institute[USGS-WiWSC]{
  USGS Wisconsin Water Science Center\\
  Middleton, WI\\[1ex]
  \texttt{wrbrooks@usgs.gov}
}
\date[May 2012]{May 3, 2012}

\begin{document}

%--- the titlepage frame -------------------------%
\begin{frame}[plain]
  \titlepage
\end{frame}

\begin{frame}{Table of Contents}
\tableofcontents
\end{frame}


\section{Overview}
%Huge amount of autmoted data collection that needs to be checked for quality assurance and control. Failures in remote sensor networks might go undetected until someone needs the data and can't get it. We need an automated method to identify changes in the data stream

\begin{frame}{Intro}

\end{frame}


\section{Past work}
%History of the theory of Bayesian surprise: it was originally developed as a mimic for human attention for use in computer vision.


\section{Current work}
%A little bit of Bayesian theory and description of python modeling


\section{Value of the work}
%Simulation study, field data


\section{Future work}
%So, so much of it.


\end{document}